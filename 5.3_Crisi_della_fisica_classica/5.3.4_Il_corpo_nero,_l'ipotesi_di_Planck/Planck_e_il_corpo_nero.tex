\documentclass[11pt]{article}

    \usepackage[breakable]{tcolorbox}
    \usepackage{parskip} % Stop auto-indenting (to mimic markdown behaviour)
    
    \usepackage{iftex}
    \ifPDFTeX
    	\usepackage[T1]{fontenc}
    	\usepackage{mathpazo}
    \else
    	\usepackage{fontspec}
    \fi

    % Basic figure setup, for now with no caption control since it's done
    % automatically by Pandoc (which extracts ![](path) syntax from Markdown).
    \usepackage{graphicx}
    % Maintain compatibility with old templates. Remove in nbconvert 6.0
    \let\Oldincludegraphics\includegraphics
    % Ensure that by default, figures have no caption (until we provide a
    % proper Figure object with a Caption API and a way to capture that
    % in the conversion process - todo).
    \usepackage{caption}
    \DeclareCaptionFormat{nocaption}{}
    \captionsetup{format=nocaption,aboveskip=0pt,belowskip=0pt}

    \usepackage[Export]{adjustbox} % Used to constrain images to a maximum size
    \adjustboxset{max size={0.9\linewidth}{0.9\paperheight}}
    \usepackage{float}
    \floatplacement{figure}{H} % forces figures to be placed at the correct location
    \usepackage{xcolor} % Allow colors to be defined
    \usepackage{enumerate} % Needed for markdown enumerations to work
    \usepackage{geometry} % Used to adjust the document margins
    \usepackage{amsmath} % Equations
    \usepackage{amssymb} % Equations
    \usepackage{textcomp} % defines textquotesingle
    % Hack from http://tex.stackexchange.com/a/47451/13684:
    \AtBeginDocument{%
        \def\PYZsq{\textquotesingle}% Upright quotes in Pygmentized code
    }
    \usepackage{upquote} % Upright quotes for verbatim code
    \usepackage{eurosym} % defines \euro
    \usepackage[mathletters]{ucs} % Extended unicode (utf-8) support
    \usepackage{fancyvrb} % verbatim replacement that allows latex
    \usepackage{grffile} % extends the file name processing of package graphics 
                         % to support a larger range
    \makeatletter % fix for grffile with XeLaTeX
    \def\Gread@@xetex#1{%
      \IfFileExists{"\Gin@base".bb}%
      {\Gread@eps{\Gin@base.bb}}%
      {\Gread@@xetex@aux#1}%
    }
    \makeatother

    % The hyperref package gives us a pdf with properly built
    % internal navigation ('pdf bookmarks' for the table of contents,
    % internal cross-reference links, web links for URLs, etc.)
    \usepackage{hyperref}
    % The default LaTeX title has an obnoxious amount of whitespace. By default,
    % titling removes some of it. It also provides customization options.
    \usepackage{titling}
    \usepackage{longtable} % longtable support required by pandoc >1.10
    \usepackage{booktabs}  % table support for pandoc > 1.12.2
    \usepackage[inline]{enumitem} % IRkernel/repr support (it uses the enumerate* environment)
    \usepackage[normalem]{ulem} % ulem is needed to support strikethroughs (\sout)
                                % normalem makes italics be italics, not underlines
    \usepackage{mathrsfs}
    

    
    % Colors for the hyperref package
    \definecolor{urlcolor}{rgb}{0,.145,.698}
    \definecolor{linkcolor}{rgb}{.71,0.21,0.01}
    \definecolor{citecolor}{rgb}{.12,.54,.11}

    % ANSI colors
    \definecolor{ansi-black}{HTML}{3E424D}
    \definecolor{ansi-black-intense}{HTML}{282C36}
    \definecolor{ansi-red}{HTML}{E75C58}
    \definecolor{ansi-red-intense}{HTML}{B22B31}
    \definecolor{ansi-green}{HTML}{00A250}
    \definecolor{ansi-green-intense}{HTML}{007427}
    \definecolor{ansi-yellow}{HTML}{DDB62B}
    \definecolor{ansi-yellow-intense}{HTML}{B27D12}
    \definecolor{ansi-blue}{HTML}{208FFB}
    \definecolor{ansi-blue-intense}{HTML}{0065CA}
    \definecolor{ansi-magenta}{HTML}{D160C4}
    \definecolor{ansi-magenta-intense}{HTML}{A03196}
    \definecolor{ansi-cyan}{HTML}{60C6C8}
    \definecolor{ansi-cyan-intense}{HTML}{258F8F}
    \definecolor{ansi-white}{HTML}{C5C1B4}
    \definecolor{ansi-white-intense}{HTML}{A1A6B2}
    \definecolor{ansi-default-inverse-fg}{HTML}{FFFFFF}
    \definecolor{ansi-default-inverse-bg}{HTML}{000000}

    % commands and environments needed by pandoc snippets
    % extracted from the output of `pandoc -s`
    \providecommand{\tightlist}{%
      \setlength{\itemsep}{0pt}\setlength{\parskip}{0pt}}
    \DefineVerbatimEnvironment{Highlighting}{Verbatim}{commandchars=\\\{\}}
    % Add ',fontsize=\small' for more characters per line
    \newenvironment{Shaded}{}{}
    \newcommand{\KeywordTok}[1]{\textcolor[rgb]{0.00,0.44,0.13}{\textbf{{#1}}}}
    \newcommand{\DataTypeTok}[1]{\textcolor[rgb]{0.56,0.13,0.00}{{#1}}}
    \newcommand{\DecValTok}[1]{\textcolor[rgb]{0.25,0.63,0.44}{{#1}}}
    \newcommand{\BaseNTok}[1]{\textcolor[rgb]{0.25,0.63,0.44}{{#1}}}
    \newcommand{\FloatTok}[1]{\textcolor[rgb]{0.25,0.63,0.44}{{#1}}}
    \newcommand{\CharTok}[1]{\textcolor[rgb]{0.25,0.44,0.63}{{#1}}}
    \newcommand{\StringTok}[1]{\textcolor[rgb]{0.25,0.44,0.63}{{#1}}}
    \newcommand{\CommentTok}[1]{\textcolor[rgb]{0.38,0.63,0.69}{\textit{{#1}}}}
    \newcommand{\OtherTok}[1]{\textcolor[rgb]{0.00,0.44,0.13}{{#1}}}
    \newcommand{\AlertTok}[1]{\textcolor[rgb]{1.00,0.00,0.00}{\textbf{{#1}}}}
    \newcommand{\FunctionTok}[1]{\textcolor[rgb]{0.02,0.16,0.49}{{#1}}}
    \newcommand{\RegionMarkerTok}[1]{{#1}}
    \newcommand{\ErrorTok}[1]{\textcolor[rgb]{1.00,0.00,0.00}{\textbf{{#1}}}}
    \newcommand{\NormalTok}[1]{{#1}}
    
    % Additional commands for more recent versions of Pandoc
    \newcommand{\ConstantTok}[1]{\textcolor[rgb]{0.53,0.00,0.00}{{#1}}}
    \newcommand{\SpecialCharTok}[1]{\textcolor[rgb]{0.25,0.44,0.63}{{#1}}}
    \newcommand{\VerbatimStringTok}[1]{\textcolor[rgb]{0.25,0.44,0.63}{{#1}}}
    \newcommand{\SpecialStringTok}[1]{\textcolor[rgb]{0.73,0.40,0.53}{{#1}}}
    \newcommand{\ImportTok}[1]{{#1}}
    \newcommand{\DocumentationTok}[1]{\textcolor[rgb]{0.73,0.13,0.13}{\textit{{#1}}}}
    \newcommand{\AnnotationTok}[1]{\textcolor[rgb]{0.38,0.63,0.69}{\textbf{\textit{{#1}}}}}
    \newcommand{\CommentVarTok}[1]{\textcolor[rgb]{0.38,0.63,0.69}{\textbf{\textit{{#1}}}}}
    \newcommand{\VariableTok}[1]{\textcolor[rgb]{0.10,0.09,0.49}{{#1}}}
    \newcommand{\ControlFlowTok}[1]{\textcolor[rgb]{0.00,0.44,0.13}{\textbf{{#1}}}}
    \newcommand{\OperatorTok}[1]{\textcolor[rgb]{0.40,0.40,0.40}{{#1}}}
    \newcommand{\BuiltInTok}[1]{{#1}}
    \newcommand{\ExtensionTok}[1]{{#1}}
    \newcommand{\PreprocessorTok}[1]{\textcolor[rgb]{0.74,0.48,0.00}{{#1}}}
    \newcommand{\AttributeTok}[1]{\textcolor[rgb]{0.49,0.56,0.16}{{#1}}}
    \newcommand{\InformationTok}[1]{\textcolor[rgb]{0.38,0.63,0.69}{\textbf{\textit{{#1}}}}}
    \newcommand{\WarningTok}[1]{\textcolor[rgb]{0.38,0.63,0.69}{\textbf{\textit{{#1}}}}}
    
    
    % Define a nice break command that doesn't care if a line doesn't already
    % exist.
    \def\br{\hspace*{\fill} \\* }
    % Math Jax compatibility definitions
    \def\gt{>}
    \def\lt{<}
    \let\Oldtex\TeX
    \let\Oldlatex\LaTeX
    \renewcommand{\TeX}{\textrm{\Oldtex}}
    \renewcommand{\LaTeX}{\textrm{\Oldlatex}}
    % Document parameters
    % Document title
    \title{Planck\_e\_il\_corpo\_nero}
    
    
    
    
    
% Pygments definitions
\makeatletter
\def\PY@reset{\let\PY@it=\relax \let\PY@bf=\relax%
    \let\PY@ul=\relax \let\PY@tc=\relax%
    \let\PY@bc=\relax \let\PY@ff=\relax}
\def\PY@tok#1{\csname PY@tok@#1\endcsname}
\def\PY@toks#1+{\ifx\relax#1\empty\else%
    \PY@tok{#1}\expandafter\PY@toks\fi}
\def\PY@do#1{\PY@bc{\PY@tc{\PY@ul{%
    \PY@it{\PY@bf{\PY@ff{#1}}}}}}}
\def\PY#1#2{\PY@reset\PY@toks#1+\relax+\PY@do{#2}}

\expandafter\def\csname PY@tok@w\endcsname{\def\PY@tc##1{\textcolor[rgb]{0.73,0.73,0.73}{##1}}}
\expandafter\def\csname PY@tok@c\endcsname{\let\PY@it=\textit\def\PY@tc##1{\textcolor[rgb]{0.25,0.50,0.50}{##1}}}
\expandafter\def\csname PY@tok@cp\endcsname{\def\PY@tc##1{\textcolor[rgb]{0.74,0.48,0.00}{##1}}}
\expandafter\def\csname PY@tok@k\endcsname{\let\PY@bf=\textbf\def\PY@tc##1{\textcolor[rgb]{0.00,0.50,0.00}{##1}}}
\expandafter\def\csname PY@tok@kp\endcsname{\def\PY@tc##1{\textcolor[rgb]{0.00,0.50,0.00}{##1}}}
\expandafter\def\csname PY@tok@kt\endcsname{\def\PY@tc##1{\textcolor[rgb]{0.69,0.00,0.25}{##1}}}
\expandafter\def\csname PY@tok@o\endcsname{\def\PY@tc##1{\textcolor[rgb]{0.40,0.40,0.40}{##1}}}
\expandafter\def\csname PY@tok@ow\endcsname{\let\PY@bf=\textbf\def\PY@tc##1{\textcolor[rgb]{0.67,0.13,1.00}{##1}}}
\expandafter\def\csname PY@tok@nb\endcsname{\def\PY@tc##1{\textcolor[rgb]{0.00,0.50,0.00}{##1}}}
\expandafter\def\csname PY@tok@nf\endcsname{\def\PY@tc##1{\textcolor[rgb]{0.00,0.00,1.00}{##1}}}
\expandafter\def\csname PY@tok@nc\endcsname{\let\PY@bf=\textbf\def\PY@tc##1{\textcolor[rgb]{0.00,0.00,1.00}{##1}}}
\expandafter\def\csname PY@tok@nn\endcsname{\let\PY@bf=\textbf\def\PY@tc##1{\textcolor[rgb]{0.00,0.00,1.00}{##1}}}
\expandafter\def\csname PY@tok@ne\endcsname{\let\PY@bf=\textbf\def\PY@tc##1{\textcolor[rgb]{0.82,0.25,0.23}{##1}}}
\expandafter\def\csname PY@tok@nv\endcsname{\def\PY@tc##1{\textcolor[rgb]{0.10,0.09,0.49}{##1}}}
\expandafter\def\csname PY@tok@no\endcsname{\def\PY@tc##1{\textcolor[rgb]{0.53,0.00,0.00}{##1}}}
\expandafter\def\csname PY@tok@nl\endcsname{\def\PY@tc##1{\textcolor[rgb]{0.63,0.63,0.00}{##1}}}
\expandafter\def\csname PY@tok@ni\endcsname{\let\PY@bf=\textbf\def\PY@tc##1{\textcolor[rgb]{0.60,0.60,0.60}{##1}}}
\expandafter\def\csname PY@tok@na\endcsname{\def\PY@tc##1{\textcolor[rgb]{0.49,0.56,0.16}{##1}}}
\expandafter\def\csname PY@tok@nt\endcsname{\let\PY@bf=\textbf\def\PY@tc##1{\textcolor[rgb]{0.00,0.50,0.00}{##1}}}
\expandafter\def\csname PY@tok@nd\endcsname{\def\PY@tc##1{\textcolor[rgb]{0.67,0.13,1.00}{##1}}}
\expandafter\def\csname PY@tok@s\endcsname{\def\PY@tc##1{\textcolor[rgb]{0.73,0.13,0.13}{##1}}}
\expandafter\def\csname PY@tok@sd\endcsname{\let\PY@it=\textit\def\PY@tc##1{\textcolor[rgb]{0.73,0.13,0.13}{##1}}}
\expandafter\def\csname PY@tok@si\endcsname{\let\PY@bf=\textbf\def\PY@tc##1{\textcolor[rgb]{0.73,0.40,0.53}{##1}}}
\expandafter\def\csname PY@tok@se\endcsname{\let\PY@bf=\textbf\def\PY@tc##1{\textcolor[rgb]{0.73,0.40,0.13}{##1}}}
\expandafter\def\csname PY@tok@sr\endcsname{\def\PY@tc##1{\textcolor[rgb]{0.73,0.40,0.53}{##1}}}
\expandafter\def\csname PY@tok@ss\endcsname{\def\PY@tc##1{\textcolor[rgb]{0.10,0.09,0.49}{##1}}}
\expandafter\def\csname PY@tok@sx\endcsname{\def\PY@tc##1{\textcolor[rgb]{0.00,0.50,0.00}{##1}}}
\expandafter\def\csname PY@tok@m\endcsname{\def\PY@tc##1{\textcolor[rgb]{0.40,0.40,0.40}{##1}}}
\expandafter\def\csname PY@tok@gh\endcsname{\let\PY@bf=\textbf\def\PY@tc##1{\textcolor[rgb]{0.00,0.00,0.50}{##1}}}
\expandafter\def\csname PY@tok@gu\endcsname{\let\PY@bf=\textbf\def\PY@tc##1{\textcolor[rgb]{0.50,0.00,0.50}{##1}}}
\expandafter\def\csname PY@tok@gd\endcsname{\def\PY@tc##1{\textcolor[rgb]{0.63,0.00,0.00}{##1}}}
\expandafter\def\csname PY@tok@gi\endcsname{\def\PY@tc##1{\textcolor[rgb]{0.00,0.63,0.00}{##1}}}
\expandafter\def\csname PY@tok@gr\endcsname{\def\PY@tc##1{\textcolor[rgb]{1.00,0.00,0.00}{##1}}}
\expandafter\def\csname PY@tok@ge\endcsname{\let\PY@it=\textit}
\expandafter\def\csname PY@tok@gs\endcsname{\let\PY@bf=\textbf}
\expandafter\def\csname PY@tok@gp\endcsname{\let\PY@bf=\textbf\def\PY@tc##1{\textcolor[rgb]{0.00,0.00,0.50}{##1}}}
\expandafter\def\csname PY@tok@go\endcsname{\def\PY@tc##1{\textcolor[rgb]{0.53,0.53,0.53}{##1}}}
\expandafter\def\csname PY@tok@gt\endcsname{\def\PY@tc##1{\textcolor[rgb]{0.00,0.27,0.87}{##1}}}
\expandafter\def\csname PY@tok@err\endcsname{\def\PY@bc##1{\setlength{\fboxsep}{0pt}\fcolorbox[rgb]{1.00,0.00,0.00}{1,1,1}{\strut ##1}}}
\expandafter\def\csname PY@tok@kc\endcsname{\let\PY@bf=\textbf\def\PY@tc##1{\textcolor[rgb]{0.00,0.50,0.00}{##1}}}
\expandafter\def\csname PY@tok@kd\endcsname{\let\PY@bf=\textbf\def\PY@tc##1{\textcolor[rgb]{0.00,0.50,0.00}{##1}}}
\expandafter\def\csname PY@tok@kn\endcsname{\let\PY@bf=\textbf\def\PY@tc##1{\textcolor[rgb]{0.00,0.50,0.00}{##1}}}
\expandafter\def\csname PY@tok@kr\endcsname{\let\PY@bf=\textbf\def\PY@tc##1{\textcolor[rgb]{0.00,0.50,0.00}{##1}}}
\expandafter\def\csname PY@tok@bp\endcsname{\def\PY@tc##1{\textcolor[rgb]{0.00,0.50,0.00}{##1}}}
\expandafter\def\csname PY@tok@fm\endcsname{\def\PY@tc##1{\textcolor[rgb]{0.00,0.00,1.00}{##1}}}
\expandafter\def\csname PY@tok@vc\endcsname{\def\PY@tc##1{\textcolor[rgb]{0.10,0.09,0.49}{##1}}}
\expandafter\def\csname PY@tok@vg\endcsname{\def\PY@tc##1{\textcolor[rgb]{0.10,0.09,0.49}{##1}}}
\expandafter\def\csname PY@tok@vi\endcsname{\def\PY@tc##1{\textcolor[rgb]{0.10,0.09,0.49}{##1}}}
\expandafter\def\csname PY@tok@vm\endcsname{\def\PY@tc##1{\textcolor[rgb]{0.10,0.09,0.49}{##1}}}
\expandafter\def\csname PY@tok@sa\endcsname{\def\PY@tc##1{\textcolor[rgb]{0.73,0.13,0.13}{##1}}}
\expandafter\def\csname PY@tok@sb\endcsname{\def\PY@tc##1{\textcolor[rgb]{0.73,0.13,0.13}{##1}}}
\expandafter\def\csname PY@tok@sc\endcsname{\def\PY@tc##1{\textcolor[rgb]{0.73,0.13,0.13}{##1}}}
\expandafter\def\csname PY@tok@dl\endcsname{\def\PY@tc##1{\textcolor[rgb]{0.73,0.13,0.13}{##1}}}
\expandafter\def\csname PY@tok@s2\endcsname{\def\PY@tc##1{\textcolor[rgb]{0.73,0.13,0.13}{##1}}}
\expandafter\def\csname PY@tok@sh\endcsname{\def\PY@tc##1{\textcolor[rgb]{0.73,0.13,0.13}{##1}}}
\expandafter\def\csname PY@tok@s1\endcsname{\def\PY@tc##1{\textcolor[rgb]{0.73,0.13,0.13}{##1}}}
\expandafter\def\csname PY@tok@mb\endcsname{\def\PY@tc##1{\textcolor[rgb]{0.40,0.40,0.40}{##1}}}
\expandafter\def\csname PY@tok@mf\endcsname{\def\PY@tc##1{\textcolor[rgb]{0.40,0.40,0.40}{##1}}}
\expandafter\def\csname PY@tok@mh\endcsname{\def\PY@tc##1{\textcolor[rgb]{0.40,0.40,0.40}{##1}}}
\expandafter\def\csname PY@tok@mi\endcsname{\def\PY@tc##1{\textcolor[rgb]{0.40,0.40,0.40}{##1}}}
\expandafter\def\csname PY@tok@il\endcsname{\def\PY@tc##1{\textcolor[rgb]{0.40,0.40,0.40}{##1}}}
\expandafter\def\csname PY@tok@mo\endcsname{\def\PY@tc##1{\textcolor[rgb]{0.40,0.40,0.40}{##1}}}
\expandafter\def\csname PY@tok@ch\endcsname{\let\PY@it=\textit\def\PY@tc##1{\textcolor[rgb]{0.25,0.50,0.50}{##1}}}
\expandafter\def\csname PY@tok@cm\endcsname{\let\PY@it=\textit\def\PY@tc##1{\textcolor[rgb]{0.25,0.50,0.50}{##1}}}
\expandafter\def\csname PY@tok@cpf\endcsname{\let\PY@it=\textit\def\PY@tc##1{\textcolor[rgb]{0.25,0.50,0.50}{##1}}}
\expandafter\def\csname PY@tok@c1\endcsname{\let\PY@it=\textit\def\PY@tc##1{\textcolor[rgb]{0.25,0.50,0.50}{##1}}}
\expandafter\def\csname PY@tok@cs\endcsname{\let\PY@it=\textit\def\PY@tc##1{\textcolor[rgb]{0.25,0.50,0.50}{##1}}}

\def\PYZbs{\char`\\}
\def\PYZus{\char`\_}
\def\PYZob{\char`\{}
\def\PYZcb{\char`\}}
\def\PYZca{\char`\^}
\def\PYZam{\char`\&}
\def\PYZlt{\char`\<}
\def\PYZgt{\char`\>}
\def\PYZsh{\char`\#}
\def\PYZpc{\char`\%}
\def\PYZdl{\char`\$}
\def\PYZhy{\char`\-}
\def\PYZsq{\char`\'}
\def\PYZdq{\char`\"}
\def\PYZti{\char`\~}
% for compatibility with earlier versions
\def\PYZat{@}
\def\PYZlb{[}
\def\PYZrb{]}
\makeatother


    % For linebreaks inside Verbatim environment from package fancyvrb. 
    \makeatletter
        \newbox\Wrappedcontinuationbox 
        \newbox\Wrappedvisiblespacebox 
        \newcommand*\Wrappedvisiblespace {\textcolor{red}{\textvisiblespace}} 
        \newcommand*\Wrappedcontinuationsymbol {\textcolor{red}{\llap{\tiny$\m@th\hookrightarrow$}}} 
        \newcommand*\Wrappedcontinuationindent {3ex } 
        \newcommand*\Wrappedafterbreak {\kern\Wrappedcontinuationindent\copy\Wrappedcontinuationbox} 
        % Take advantage of the already applied Pygments mark-up to insert 
        % potential linebreaks for TeX processing. 
        %        {, <, #, %, $, ' and ": go to next line. 
        %        _, }, ^, &, >, - and ~: stay at end of broken line. 
        % Use of \textquotesingle for straight quote. 
        \newcommand*\Wrappedbreaksatspecials {% 
            \def\PYGZus{\discretionary{\char`\_}{\Wrappedafterbreak}{\char`\_}}% 
            \def\PYGZob{\discretionary{}{\Wrappedafterbreak\char`\{}{\char`\{}}% 
            \def\PYGZcb{\discretionary{\char`\}}{\Wrappedafterbreak}{\char`\}}}% 
            \def\PYGZca{\discretionary{\char`\^}{\Wrappedafterbreak}{\char`\^}}% 
            \def\PYGZam{\discretionary{\char`\&}{\Wrappedafterbreak}{\char`\&}}% 
            \def\PYGZlt{\discretionary{}{\Wrappedafterbreak\char`\<}{\char`\<}}% 
            \def\PYGZgt{\discretionary{\char`\>}{\Wrappedafterbreak}{\char`\>}}% 
            \def\PYGZsh{\discretionary{}{\Wrappedafterbreak\char`\#}{\char`\#}}% 
            \def\PYGZpc{\discretionary{}{\Wrappedafterbreak\char`\%}{\char`\%}}% 
            \def\PYGZdl{\discretionary{}{\Wrappedafterbreak\char`\$}{\char`\$}}% 
            \def\PYGZhy{\discretionary{\char`\-}{\Wrappedafterbreak}{\char`\-}}% 
            \def\PYGZsq{\discretionary{}{\Wrappedafterbreak\textquotesingle}{\textquotesingle}}% 
            \def\PYGZdq{\discretionary{}{\Wrappedafterbreak\char`\"}{\char`\"}}% 
            \def\PYGZti{\discretionary{\char`\~}{\Wrappedafterbreak}{\char`\~}}% 
        } 
        % Some characters . , ; ? ! / are not pygmentized. 
        % This macro makes them "active" and they will insert potential linebreaks 
        \newcommand*\Wrappedbreaksatpunct {% 
            \lccode`\~`\.\lowercase{\def~}{\discretionary{\hbox{\char`\.}}{\Wrappedafterbreak}{\hbox{\char`\.}}}% 
            \lccode`\~`\,\lowercase{\def~}{\discretionary{\hbox{\char`\,}}{\Wrappedafterbreak}{\hbox{\char`\,}}}% 
            \lccode`\~`\;\lowercase{\def~}{\discretionary{\hbox{\char`\;}}{\Wrappedafterbreak}{\hbox{\char`\;}}}% 
            \lccode`\~`\:\lowercase{\def~}{\discretionary{\hbox{\char`\:}}{\Wrappedafterbreak}{\hbox{\char`\:}}}% 
            \lccode`\~`\?\lowercase{\def~}{\discretionary{\hbox{\char`\?}}{\Wrappedafterbreak}{\hbox{\char`\?}}}% 
            \lccode`\~`\!\lowercase{\def~}{\discretionary{\hbox{\char`\!}}{\Wrappedafterbreak}{\hbox{\char`\!}}}% 
            \lccode`\~`\/\lowercase{\def~}{\discretionary{\hbox{\char`\/}}{\Wrappedafterbreak}{\hbox{\char`\/}}}% 
            \catcode`\.\active
            \catcode`\,\active 
            \catcode`\;\active
            \catcode`\:\active
            \catcode`\?\active
            \catcode`\!\active
            \catcode`\/\active 
            \lccode`\~`\~ 	
        }
    \makeatother

    \let\OriginalVerbatim=\Verbatim
    \makeatletter
    \renewcommand{\Verbatim}[1][1]{%
        %\parskip\z@skip
        \sbox\Wrappedcontinuationbox {\Wrappedcontinuationsymbol}%
        \sbox\Wrappedvisiblespacebox {\FV@SetupFont\Wrappedvisiblespace}%
        \def\FancyVerbFormatLine ##1{\hsize\linewidth
            \vtop{\raggedright\hyphenpenalty\z@\exhyphenpenalty\z@
                \doublehyphendemerits\z@\finalhyphendemerits\z@
                \strut ##1\strut}%
        }%
        % If the linebreak is at a space, the latter will be displayed as visible
        % space at end of first line, and a continuation symbol starts next line.
        % Stretch/shrink are however usually zero for typewriter font.
        \def\FV@Space {%
            \nobreak\hskip\z@ plus\fontdimen3\font minus\fontdimen4\font
            \discretionary{\copy\Wrappedvisiblespacebox}{\Wrappedafterbreak}
            {\kern\fontdimen2\font}%
        }%
        
        % Allow breaks at special characters using \PYG... macros.
        \Wrappedbreaksatspecials
        % Breaks at punctuation characters . , ; ? ! and / need catcode=\active 	
        \OriginalVerbatim[#1,codes*=\Wrappedbreaksatpunct]%
    }
    \makeatother

    % Exact colors from NB
    \definecolor{incolor}{HTML}{303F9F}
    \definecolor{outcolor}{HTML}{D84315}
    \definecolor{cellborder}{HTML}{CFCFCF}
    \definecolor{cellbackground}{HTML}{F7F7F7}
    
    % prompt
    \makeatletter
    \newcommand{\boxspacing}{\kern\kvtcb@left@rule\kern\kvtcb@boxsep}
    \makeatother
    \newcommand{\prompt}[4]{
        \ttfamily\llap{{\color{#2}[#3]:\hspace{3pt}#4}}\vspace{-\baselineskip}
    }
    

    
    % Prevent overflowing lines due to hard-to-break entities
    \sloppy 
    % Setup hyperref package
    \hypersetup{
      breaklinks=true,  % so long urls are correctly broken across lines
      colorlinks=true,
      urlcolor=urlcolor,
      linkcolor=linkcolor,
      citecolor=citecolor,
      }
    % Slightly bigger margins than the latex defaults
    
    \geometry{verbose,tmargin=1in,bmargin=1in,lmargin=1in,rmargin=1in}
    
    

\begin{document}
    
    \maketitle
    
    

    
    Table of Contents{}

{{1~~}Introduzione al problema del corpo nero }

{{1.1~~}Un po' di storia.. }

{{1.2~~}Kirchhoff }

{{1.3~~}legge di Rayleigh-Jeans }

{{1.3.0.1~~}La legge di Wien}

{{1.3.0.2~~}Legge di stefan-Boltzmann}

{{2~~}Dimostrazione della formula di Planck}

{{2.1~~}Ipotesi}

{{2.1.1~~}Ipotesi di quantizzazione - Planck}

{{2.1.2~~}Fattore di Boltzmann}

{{2.1.2.1~~}\(h\nu \ll k_BT\)}

{{2.1.2.2~~}\(h\nu=k_BT\)}

{{2.1.2.3~~}\(h\nu \gt k_BT\)}

{{2.2~~}L'equilibrio termodinamico ed energia del corpo nero}

{{2.2.1~~}Studiamo il denominatore}

{{2.2.2~~}Studiamo il numeraore}

{{2.3~~}L'energia media e la distribuzione di Planck}

{{2.4~~}la densità di energia - distribuzione di Planck}

{{3~~}Conseguenze della formula di Planck}

{{3.1~~}La legge di Wien}

{{3.2~~}La legge di Stefan}

{{3.3~~}La legge di Rayleigh-Jeans}

    \begin{tcolorbox}[breakable, size=fbox, boxrule=1pt, pad at break*=1mm,colback=cellbackground, colframe=cellborder]
\prompt{In}{incolor}{1}{\boxspacing}
\begin{Verbatim}[commandchars=\\\{\}]
\PY{k+kn}{import} \PY{n+nn}{numpy} \PY{k}{as} \PY{n+nn}{np}
\PY{k+kn}{import} \PY{n+nn}{matplotlib}
\PY{k+kn}{import} \PY{n+nn}{matplotlib}\PY{n+nn}{.}\PY{n+nn}{pyplot} \PY{k}{as} \PY{n+nn}{plt}
\PY{k+kn}{from} \PY{n+nn}{IPython}\PY{n+nn}{.}\PY{n+nn}{display} \PY{k+kn}{import} \PY{n}{Image}
\PY{k+kn}{from} \PY{n+nn}{IPython}\PY{n+nn}{.}\PY{n+nn}{display} \PY{k+kn}{import} \PY{n}{IFrame}
\PY{k+kn}{from} \PY{n+nn}{scipy}\PY{n+nn}{.}\PY{n+nn}{constants} \PY{k+kn}{import} \PY{n}{h}\PY{p}{,} \PY{n}{c}\PY{p}{,} \PY{n}{k}\PY{p}{,} \PY{n}{pi}
\end{Verbatim}
\end{tcolorbox}

    \hypertarget{la-legge-di-planck}{%
\section{La legge di Planck}\label{la-legge-di-planck}}

\[
u(\nu,T)= \frac{8\pi \nu^2}{c^3}  \frac{h\nu}{e^{\frac{h\nu}{kT}}-1}
\]

    \hypertarget{introduzione-al-problema-del-corpo-nero}{%
\subsection{Introduzione al problema del corpo nero
}\label{introduzione-al-problema-del-corpo-nero}}

\hypertarget{un-po-di-storia..}{%
\subsubsection{Un po' di storia.. }\label{un-po-di-storia..}}

Tutti noi abbiamo esperienza del fatto che una brace gialla in un camino
ardente \((1400-1600\, °C)\) risulta più calda di una brace rossa di un
camino in via di spegnimento \((1200-1300\, °C)\). All'aumentare della
temperatura, il colore della luce emessa si sposta dal rosso verso il
blu. La correlazione tra il colore della radiazione termica emessa e la
temperatura di oggetti incandescenti, incuriosì l'astronomo
anglo-tedesco William Herschel che, già nel 1800 (prima dell'insorgere
delle necessità industriali di cui si è appena parlato), effettuò una
scoperta sorprendente. Facendo passare la luce del Sole attraverso un
prisma e, facendo in modo che ognuno dei colori in cui veniva scomposta
la luce andasse a colpire uno solo dei diversi termometri disposti su di
un banco, si accorse che i termometri non indicavano tutti la stessa
temperatura. Si tratta della prima evidenza del fatto che l'intensità
della radiazione che proviene dal Sole è diversa per ognuna delle
lunghezze d'onda incidenti. In particolare, la \textbf{distribuzione
spettrale della radiazione emessa dal Sole, assomiglia moltissimo a
quella del già citato corpo nero.}

In realtà, ogni corpo che non si trovi in equilibrio termico con
l'ambiente che lo circonda, emette radiazione in forma di comune calore.
Il fenomeno è riconducibile all'oscillazione delle cariche del corpo
stesso che, in accordo con la teoria di Maxwell (e le osservazioni di
Hertz) muovendosi emettono radiazione, rallentando. In questo modo il
corpo si raffredda.

\hypertarget{kirchhoff}{%
\subsubsection{Kirchhoff }\label{kirchhoff}}

Planck riconoscerà l'importanza del modello di Kirchhoff per lo sviluppo
della sua teoria, affermando: \emph{Kirchhoff ha dimostrato che lo
stato della radiazione termica all'interno di una cavità delimitata da
una sostanza, di qualsiasi natura, che la assorbe e la emette, a una
temperatura uniforme, è totalmente indipendente dalla natura della
sostanza stessa. È stata così dimostrata l'esistenza di una funzione
universale che dipende solo dalla temperatura e dalla lunghezza d'onda
(come dalla frequenza), ma in nessun modo dalle proprietà di alcuna
sostanza. La scoperta di tale funzione straordinaria prometteva una
comprensione più profonda della relazione tra energia e temperatura che
è di fatto,  il problema più importante della termodinamica e, di conseguenza, di tutta la fisica molecolare.}

\hypertarget{legge-di-rayleigh-jeans}{%
\subsubsection{legge di Rayleigh-Jeans }\label{legge-di-rayleigh-jeans}}

Lord Rayleigh vuole ricavare la funzione che descrive l'energia per
unità di volume e di lunghezza d'onda che viene emessa all'interno del
corpo nero posto ad una temperatura \(T\). Ossia vuole ricavare \[
I(\lambda,T)
\] Sapendo che valgono le leggi di Wien e di Stefan-Boltzmann

\begin{center}\rule{0.5\linewidth}{0.5pt}\end{center}

\hypertarget{la-legge-di-wien}{%
\subparagraph{La legge di Wien}\label{la-legge-di-wien}}

La legge dello spostamento di Wien è: \[
\lambda_{max}T=b; \quad b\,\text{è una costante misurata empiricamente, detta costante di Wien}\\
b=2,898\cdot 10^{-3}\, m\cdot K
\] \#\#\#\#\# Legge di stefan-Boltzmann La potenza per unità di area
totale irraggiata da un corpo che si trova ad una data temperatura è: \[
I_{tot}(T)=\sigma\cdot T^4; \quad \sigma=5.67\cdot10^{-8}\, \frac{W}{m^2K^4}
\]

\begin{center}\rule{0.5\linewidth}{0.5pt}\end{center}

Si tratta, comunque, di un approccio di tipo classico, basato sulla
possibilità di \textbf{assorbire ed emettere radiazione in cavità (ad
ogni lunghezza d'onda) , solo per mezzo di onde elettromagnetiche
stazionarie.}

Per onda stazionaria s'intende un'onda che non si propaga lungo una
fissata direzione spaziale, ma oscilla soltanto nel tempo. Essa
presenterà dunque dei punti in cui è fissa e di ampiezza nulla, detti
nodi.

Nel modello di Rayleigh: 1. i nodi delle onde stazionarie si trovano
sempre sulle pareti della cavità 2. ogni trasmissione di energia
all'interno della cavità è vincolato ad avvenire per mezzo di un onda
stazionaria.

Su questa base, egli associò ogni onda a un oscillatore armonico
unidimensionale (l'oscillazione armonica è legata all'oscillazione del
campo elettrico). In questo modo, la densità spettrale d'energia
\(u_\nu(T)\), ovvero l'energia per unità di volume in un intervallo
infinitesimo di lunghezza d'onda, sarà ottenuta come prodotto tra la
densità di volume \(n_\lambda\) dei modi di lunghezza d'onda compresa
tra \(\lambda\) e \(\lambda+d\lambda\) e l'energia media
\(<\epsilon_\lambda>\) di un oscillatore, ovvero: \[
\rho=n_\lambda<\epsilon_\lambda>.
\]

Utilizzando il principio di equipartizione e l'ipotesi di un campo
elettromagnetico formato da onde stazionari Raylegh arriva alla
conclusione che La densità di energia \(u_\nu\) per unità di intervallo
di frequenza a una frequenza \(\nu\) è, secondo The Rayleigh-Jeans
Radiation,

\[
u_\nu(T) = \frac{8\pi}{c^2}\nu^2k_bT
\]

Per un dimostrazione
\href{https://github.com/massimobosetti/Physics-Problem/blob/master/5_anno/5.3\%20Crisi\%20della\%20fisica\%20classica/5.3.6.2\%20Derivazione\%20della\%20legge\%20di\%20Rayleigh-Jeans.ipynb}{legge
di Rayleigh-Jeans}

Che porta al problema della catastrofe ultravioletta e la mancanza di
accordo con i dati sperimentali a frequenze elvate.

    \hypertarget{dimostrazione-della-formula-di-planck}{%
\subsection{Dimostrazione della formula di
Planck}\label{dimostrazione-della-formula-di-planck}}

\hypertarget{ipotesi}{%
\subsubsection{Ipotesi}\label{ipotesi}}

\hypertarget{ipotesi-di-quantizzazione---planck}{%
\paragraph{Ipotesi di quantizzazione -
Planck}\label{ipotesi-di-quantizzazione---planck}}

L'energia del campo magnetico all'interno della cavità, all'equilibrio,
non può prendere qualsiasi valore, ma può assumere solo multipli interi
di una energia \emph{discretizzata}, un \textbf{quanto di energia} che
dipende dalla frequenza e da una costante, cioè:

\[
\label{eq:planck}
E_n=n\,h\nu \quad\qquad (n=0,1,2,\cdots)\,\in \mathbb{N}
\]

    \hypertarget{fattore-di-boltzmann}{%
\paragraph{Fattore di Boltzmann}\label{fattore-di-boltzmann}}

Le energie vanno ``pesate'' con un fattore. Non tutte le energia hanno
lo stesso peso. Anzi andando ad energie elevate, ovvero frequenze
elevate, queste debbono necessariamente essere meno probabile. Il
fattore che media è il fattore di Boltzmann \[
f(E_n)=e^{-\frac{E_n}{k_BT}}
\]

come ben si può vedere dal grafico, questo fattore va a pesare in modo
differente energie diverse e fornisce la probaibilità che in uno sistema
ad una temperatura \(T\) ci sia un oscillatore, o un modo del campo
elettromagnetico, con energia \(h\nu\).

    \hypertarget{hnu-ll-k_bt}{%
\subparagraph{\texorpdfstring{\(h\nu \ll k_BT\)}{h\textbackslash{}nu \textbackslash{}ll k\_BT}}\label{hnu-ll-k_bt}}

Per capire il grafico fissiamo una temperatura \(T\) e pensiamo a 3
regimi diversi:

Abbiamo quindi una suddivisione delle energie in molti intervalli
piccoli

    \hypertarget{hnuk_bt}{%
\subparagraph{\texorpdfstring{\(h\nu=k_BT\)}{h\textbackslash{}nu=k\_BT}}\label{hnuk_bt}}

Come si può notare se andiamo ad una energia per cui \(k_BT=h\nu\)
allora il fattore si

    \hypertarget{hnu-gt-k_bt}{%
\subparagraph{\texorpdfstring{\(h\nu \gt k_BT\)}{h\textbackslash{}nu \textbackslash{}gt k\_BT}}\label{hnu-gt-k_bt}}

    \hypertarget{lequilibrio-termodinamico-ed-energia-del-corpo-nero}{%
\subsubsection{L'equilibrio termodinamico ed energia del corpo
nero}\label{lequilibrio-termodinamico-ed-energia-del-corpo-nero}}

All'equilibrio termodinamico vogliamo conoscere l'energia media \(<E>\),
quindi:

L'energia media è data da:

\begin{equation}\label{eq:ener_media}
<E> = \frac{\text{somma delle energie di tutti gli stati nella cavità}\times\text{peso di ogni energia}}{\text{numero totale degli stati}\times\text{peso di ogni energia}}
\end{equation}

    Ricordando che: 1. \(E_n=nh\nu;\) energia dello stato 2.
\(f(E_n)=e^{-\frac{E_n}{k_BT}};\) peso dello stano \(n\)-esimo

    Otteniamo che:

\begin{equation}\label{eq:energia media}
<E>=\frac{\sum_{n}E_n\cdot f(E_n)}{\sum_{n} f(E_n)}=\frac{\sum_{n}E_n\cdot e^{-\frac{E_n}{k_BT}}}{\sum_{n} e^{-\frac{E_n}{k_BT}}}\\
\end{equation}

    \[
\Downarrow
\]

\begin{equation}\label{eq:energia media1}
    \frac{\sum_{n}nh\nu\cdot e^{-\frac{nh\nu}{k_BT}}}{\sum_{n} e^{-\frac{nh\nu}{k_BT}}}
\end{equation}

    \textbf{Dobbiamo quindi capire come fare questa somma infinita che va da
\(n=0\) a \(n=\infty\). cioè il rapporto di due serie.}

Innanzitutto facciamo una sotituzione per rendere il problema più
leggibile

\begin{equation}\label{eq:energia media_1}
 x=\frac{h\nu}{kT}\\
 \, \\
 \Downarrow \\
 \, \\
 <E> = \frac{\sum_{n}nkTx\cdot e^{-nx}}{\sum_{n}  e^{-nx}}
\end{equation}

    Studiamo in modo separato il denominatore ed il numeratore

\hypertarget{studiamo-il-denominatore}{%
\paragraph{Studiamo il denominatore}\label{studiamo-il-denominatore}}

Introduciamo la funzione

\begin{equation}\label{eq:Z}
Z(x) = \sum_{n=0}^\infty  e^{-nx}
\end{equation}

    come si può sosservare è un a serie geometrica

\begin{equation}\label{eq:ser geom}
1+e^{-x}+e^{-2x}+e^{-3x}+\dots +e^{-nx}+\dots
\end{equation}

    Con ragione \[
q=e^{-x} \Rightarrow \text{ di somma } \ \sum_{n=0}^\infty q^n = \frac{1}{1-q}\\
\]

    \[
\Downarrow\\
Z(x) = \sum_{n=0}^\infty  e^{-nx} =\frac{1}{1-e^{-x}}
\]

    \hypertarget{studiamo-il-numeraore}{%
\paragraph{Studiamo il numeraore}\label{studiamo-il-numeraore}}

Il numeratore è \begin{equation}\label{eq:num_p}
kT\sum_{n=0}^\infty n x e^{-nx}
\end{equation}

    Ritorniamo alla funzione \(Z(x)= \sum_{n=0}^\infty e^{-nx}\) e proviamo
a ricavare la derivata.

\begin{equation}\label{eq:der_Z}
\frac{dZ(x)}{dx}= \frac{d}{dx}\sum_{n=0}^\infty  e^{-nx}
\end{equation}

    Abbiamo qui la derivata si una somma, e per quanto infinita, le regole
di derivazione (visto che le funzioni sono continue e derivabili in
tutto \(\mathbb{R}\)) valgono ancora, e quindi la derivata della somma è
la somma delle derivate:

\begin{equation}\label{eq:der_Z1}
\frac{dZ(x)}{dx}=- \sum_{n=0}^\infty n e^{-nx}
\end{equation}

    L'osservazione da fare è che quindi il numeratore e la derivata di
\(Z(x)\) sono uguali a meno di un fattore, cioè:

\begin{equation}\label{eq:num_p1}
\sum_{n=0}^\infty nx e^{-nx} = - x \frac{dZ(x)}{dx}
\end{equation}

    \hypertarget{lenergia-media-e-la-distribuzione-di-planck}{%
\subsubsection{L'energia media e la distribuzione di
Planck}\label{lenergia-media-e-la-distribuzione-di-planck}}

Sosttiutiamo quindi i risultati ottenuti

\[
<E> = \frac{\sum_{n}nkTx\cdot e^{-nx}}{\sum_{n}  e^{-nx}} = \frac{-kTx  \frac{dZ(x)}{dx}}{Z(x)}
\] Guardiamo bene il termine

    \[
\frac{\frac{dZ(x)}{dx}}{Z(x)} = \frac{1}{Z(x)}\frac{dZ(x)}{dx}
\]

    Dovreste accorgervi che se facciamo la derivata di \[
\frac{d\ln{f(x)}}{dx} = \frac{1}{f(x)}\frac{df(x)}{dx}
\]

    E quindi \[
 \frac{1}{Z(x)}\frac{dZ(x)}{dx} = \frac{d\ln{Z(x)}}{dx} 
\]

    Che sostituita nel calcolo dell'energia media da: \[
 <E> = -kTx\frac{1}{Z(x)}\frac{dZ(x)}{dx}  = -kTx \frac{d\ln{Z(x)}}{dx}
\]

    \[
\Downarrow\\
\, \\
 -kTx \frac{d\left (\ln{\left ( 1-e^{-x}\right )^{-1}}\right )}{dx} = -kTx \frac{d\left (-\ln{\left ( 1-e^{-x}\right )}\right )}{dx}
\]

    \[
\Downarrow
\\
\, \\
kTx \frac{d\left (\ln{\left ( 1-e^{-x}\right )}\right )}{dx} =  kTx \frac{e^{-x}}{ 1-e^{-x}} 
\]

    Infine operando l'ultima sotituzione otteniamo la \textbf{Legge di
Planck} in funzione della frequenza della frequenza:

\[
<E> = h\nu \frac{e^{-\frac{h\nu}{kT}}}{1-e^{-\frac{h\nu}{kT}}} =  \frac{h\nu}{e^{\frac{h\nu}{kT}}-1};\quad\quad\quad \text{Legge di Planck}
\] Si noti che \[
\nu=c / \lambda \Rightarrow d \nu=-\left(c / \lambda^{2}\right) d \lambda \Rightarrow \frac{d \nu}{d \lambda}=-\frac{c}{\lambda^{2}}
\]

    \hypertarget{la-densituxe0-di-energia---distribuzione-di-planck}{%
\subsubsection{la densità di energia - distribuzione di
Planck}\label{la-densituxe0-di-energia---distribuzione-di-planck}}

Per ottenere la densità energetica dobbiamo moltiplicare per la densità
degli stati, euqivalente a quelli di Rayleghi-Jeans $dN =
\frac{8\pi L^3\nu^2}{c^3}d\nu $ e dividere per il volume \(L^3\), \[
u(\nu,T)= \frac{8\pi \nu^2}{c^3}  \frac{h\nu}{e^{\frac{h\nu}{kT}}-1}
\] che è in ottimo accordo con i dati sperimentali.

    Per ricavarer la densità in funzione della lunghezza d'onda si noti che

\begin{equation}\label{eq:PLANCK}
\nu=c / \lambda \Rightarrow d \nu=-\left(c / \lambda^{2}\right) d \lambda \Rightarrow \frac{d \nu}{d \lambda}=-\frac{c}{\lambda^{2}}
\end{equation}

quindi la formula di Planck eq. (20) per la distribuzione spettrale
della radiazione espressa in lunghezza d'onda si scrive \[
u_{T}(\lambda) d \lambda=\frac{8 \pi h c}{\lambda^{5}} \frac{d \lambda}{e^{k / \lambda k_{n} T}-1}
\]

    \begin{tcolorbox}[breakable, size=fbox, boxrule=1pt, pad at break*=1mm,colback=cellbackground, colframe=cellborder]
\prompt{In}{incolor}{2}{\boxspacing}
\begin{Verbatim}[commandchars=\\\{\}]
\PY{n}{IFrame}\PY{p}{(}\PY{n}{src}\PY{o}{=}\PY{l+s+s1}{\PYZsq{}}\PY{l+s+s1}{https://phet.colorado.edu/sims/html/blackbody\PYZhy{}spectrum/latest/blackbody\PYZhy{}spectrum\PYZus{}en.html}\PY{l+s+s1}{\PYZsq{}}\PY{p}{,} \PY{n}{width}\PY{o}{=}\PY{l+m+mi}{1000}\PY{p}{,} \PY{n}{height}\PY{o}{=}\PY{l+m+mi}{800}\PY{p}{)}
\end{Verbatim}
\end{tcolorbox}

            \begin{tcolorbox}[breakable, size=fbox, boxrule=.5pt, pad at break*=1mm, opacityfill=0]
\prompt{Out}{outcolor}{2}{\boxspacing}
\begin{Verbatim}[commandchars=\\\{\}]
<IPython.lib.display.IFrame at 0x7fca49152650>
\end{Verbatim}
\end{tcolorbox}
        
    \hypertarget{conseguenze-della-formula-di-planck}{%
\subsection{Conseguenze della formula di
Planck}\label{conseguenze-della-formula-di-planck}}

Utilizando leq.(26) è possibile ricavare la legge di Wien, di Stefan e
la legge di Rayleigh-Jeans come limite per \(\nu \rightarrow 0\)

\hypertarget{la-legge-di-wien}{%
\subsubsection{La legge di Wien}\label{la-legge-di-wien}}

Imponendo che la derivata rispetto a $ \nu$ dell'eq. (26) si annulli
si determina $\nu_{\max }$ e $\lambda_{\max }=c / \nu_{\max }$

\[
\begin{aligned}
\frac{d \rho_{T}(\nu)}{d \nu}=0 & \Rightarrow\left(3 \nu^{2}-\nu^{3} \frac{h}{k_{B} T} e^{\omega_{W} / k_{B} T}\left(e^{k w / k_{B} T}-1\right)^{-1}\right)=0 \\
& \Rightarrow 3\left(e^{\alpha}-1\right)-\alpha e^{\alpha}=0 \Rightarrow e^{\alpha}(1-\alpha / 3)=1
\end{aligned}
\]

Per trovare la soluzione dell'eq.(24) occorre risolvere l'equazione
trascendente \[
e^{-\alpha}=(1-\alpha / 3) \Rightarrow \alpha \sim 2.8 \Rightarrow \lambda_{\max } T=0.29\, \mathrm{cm} K
\]

\begin{itemize}
\tightlist
\item
  La soluzione dell'equazione si trova graficamente cercando il valore
  di \(\alpha\) per il quale la retta \(1-\alpha / 3\) intercetta la
  curva esponenziale \(e^{-\alpha} .\) II valore \(\alpha=0\) va
  scartato perché corrisponde ad \(\nu=0\) (minimo della curva).
\end{itemize}

    

    \hypertarget{la-legge-di-stefan}{%
\subsubsection{La legge di Stefan}\label{la-legge-di-stefan}}

Calcoliamo la radianza usando l'eq.( 26 ), ricordando che la radianza
spettrale è proporzionale alla densità d'energia \[
u_{T} \propto \int_{0}^{\infty} \rho_{T}(\nu) d \nu=\int_{0}^{\infty} \frac{8 \pi \nu^{2}}{c^{3}} \frac{h \nu}{e^{k \omega / k_{B} T}-1} d \nu
\]

Cambiando variabile
\[\alpha=h \nu / k_{B} T \Rightarrow k_{B} T / h d \alpha=d \nu\]

L'eq.(33) si scrive \[
u_{T} \propto \frac{8 \pi h}{c^{3}}\left(\frac{k_{B} T}{h}\right)^{4} \int_{0}^{\infty} \frac{\alpha^{3} e^{-\alpha} d \alpha}{1-e^{-\alpha}}
\]

Leq.(33) mostra una proporzionalità tra la radianza e la quarta potenza
della temperatura, in quanto lintegrale indefinito è un numero. Tale
integrale si può esplicitamente calcolare mediante uno sviluppo in serie
di potenze del denominatore dell'integrando (che non facciamo ma che da
come risultato:

\[
\int_{0}^{\infty} \frac{\alpha^{3} e^{-\alpha} d \alpha}{1-e^{-\alpha}}= \frac{\pi^{4}}{15}\]

Inserendo tale espressione l'integrale in \(\alpha\) dell'eq.(33)
diventa \[
u_{T}  \propto \frac{8 \pi h}{c^{3}}\left(\frac{k_{B} T}{h}\right)^{4}  \frac{\pi^{4}}{15}
\]

    \hypertarget{la-legge-di-rayleigh-jeans}{%
\subsubsection{La legge di
Rayleigh-Jeans}\label{la-legge-di-rayleigh-jeans}}

Se \(h \nu<<k_{B} T\) possinmo sviluppare lesponenzinle che appare nel
denominntore dell'eq.(26) e fermarci al primo ordine \[
u_{T}(\nu)_{\nu \rightarrow 0} \sim k_{B} T \frac{8 \pi \nu^{2}}{c^{3}}
\]

che è la legge classica vedi di Rayleigh-Jeans. Si ricordi che

\[
e^{x}=\sum_{n=0}^{\infty} \frac{x^{n}}{n !}=1+x+x^{2} / 2+x^{3} / 6 \ldots
\] Si noti che la disuguaglianza \(h \nu<<k_{B} T\) é verificata a basse
frequenze e per ogni valore di \(\nu\) se \(h \rightarrow 0,\) limite
classico.

    \begin{tcolorbox}[breakable, size=fbox, boxrule=1pt, pad at break*=1mm,colback=cellbackground, colframe=cellborder]
\prompt{In}{incolor}{ }{\boxspacing}
\begin{Verbatim}[commandchars=\\\{\}]

\end{Verbatim}
\end{tcolorbox}


    % Add a bibliography block to the postdoc
    
    
    
\end{document}
